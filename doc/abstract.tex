\begin{abstract}
The advantage of adding modern day information and communication technology (ICT) in the supervisory control and data acquisition (SCADA) system associated with the power network comes at the cost of an increased risk due to cyber intrusion. A well planned malicious attack on the SCADA system can not only compromise the communication network, but also cause catastrophic effects on the power grid in form of a widespread blackout. In this context, a lot of prior work deals with the comprehensive modeling of the cyber-physical system (CPS) and evaluating the possible vulnerabilities. In the present work, a Bayesian attack tree based approach is used to model cyber attacks in the SCADA network and the associated risk is evaluated as the combined effect on the communication and power system. This avoids the detailed modeling of every component in the CPS and considers only the critical vulnerabilities required to be exploited to perform the attack. Furthermore, the model takes into account the skill level of the adversary and the difficulty in intruding through each type of vulnerability. The proposed cyber attack model is applied on the IEEE-39 bus system with an associated SCADA network. The risk of a cyber attack on the critical vulnerabilities is evaluated for the power system.
\end{abstract}

\begin{IEEEkeywords}
cyber-physical security, Bayesian attack tree, mean time to compromise, vulnerability assessment
\end{IEEEkeywords}