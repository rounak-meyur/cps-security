\section{Conclusion}\label{sec:conclude}
A Bayesian attack tree approach has been used to model cyber attacks in the SCADA networks of substations and control centers. The attack trees are identified based on the probable intrusion paths in each SCADA system with a goal to gain administrative access in control assets of the network. The vulnerabilities in each attack path and probability of their successful exploit are identified based on the latest vulnerability statistics in cyber systems. The time to compromise vulnerabilities is calculated for each of them and considered as a metric of intruder attack efficiency. The risk of a cyber attack is assessed from the attack efficiency as well as the impact on power grid.

In this work, each vulnerability is considered to be distinct from one another. In reality, some vulnerabilities are common in all systems. If the exploit code is available to the adversary, time to compromise a known vulnerability reduces significantly. This modification to the evaluation of time to compromise can be considered as a scope of future work. A similar approach can be used to model man-in-the-middle attacks using flooding of queues which can be another possible scope of future work. 